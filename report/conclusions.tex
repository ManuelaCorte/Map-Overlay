\section{Conclusions}
\subsection{Possible Improvements}
The implementation of the overlay algorithm is functional and produces the expected results. However, there are several improvements that could be made to the implementation to make it more efficient and robust. Some of the possible improvements are:
\begin{itemize}
    \item \textbf{Status data structure}: the current implementation uses a list to store the status. This is inefficient because the list has to be sorted at each iteration. Initially, the implementation used a Red-Black Tree for the status but updating the position of the segments intersections line basically meant reconstructing the entire tree at each iteration. A different data structure that allows to update the value of a node (i.e. the intersection x-coordinate of a segment with the sweep line) without removing and reinserting the node might work better.
    \item \textbf{Holes in faces}: the current implementation of DCELs doesn't support holes in faces which would make it more general and versatile. This wasn't done due to time constraints.
    \item \textbf{More tests for overlay}: while I was able to find a variety of tests to check the correctness of the intersection algorithm, the overlay algorithm could benefit from more and more complex tests. Moreover, the tests only check the correctness of the algorithm by comparing the number of faces in the output DCEL with the expected number but this doesn't necessarily cover all the possible errors that could occur.
\end{itemize}
